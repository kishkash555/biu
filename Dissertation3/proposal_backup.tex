%%%%%%%%%%%%%%%%%%%%%%%%%%%%%%%%%%%%%%%%%%%%%%%%%%%%%%%%%%%%%%%%%%%%%%%%%%%%%%%%
%2345678901234567890123456789012345678901234567890123456789012345678901234567890
%        1         2         3         4         5         6         7         8

%\documentclass[letterpaper, 10 pt, conference]{ieeeconf}  % Comment this line out
                                                          % if you need a4paper
\documentclass[a4paper, 11pt]{article}      % Use this line for a4
                                                          % paper

\usepackage[utf8]{inputenc}
\usepackage[T1]{fontenc}

% The following packages can be found on http:\\www.ctan.org
%\usepackage{graphics} % for pdf, bitmapped graphics files
%\usepackage{epsfig} % for postscript graphics files
%\usepackage{mathptmx} % assumes new font selection scheme installed
%\usepackage{mathptmx} % assumes new font selection scheme installed
%\usepackage{amsmath} % assumes amsmath package installed
%\usepackage{amssymb}  % assumes amsmath package installed

\title{\LARGE \bf
Team performance and physiological synchrony
}

\author{Shahar Siegman}

\begin{document}



\maketitle
\thispagestyle{empty}
\pagestyle{empty}


%%%%%%%%%%%%%%%%%%%%%%%%%%%%%%%%%%%%%%%%%%%%%%%%%%%%%%%%%%%%%%%%%%%%%%%%%%%%%%%%
\begin{abstract}

A research proposal.

\end{abstract}

\section{Motivation - team performance}
Team performance is an important research topic in social sciences. Successful teams are characterized by both higher performance and better individual (subjective) experiences during the interaction, compared to less successful teams. This is why it's not surprising that team success is determined not only by individual competence levels, but also by the extent to which team members are able to respond effectively to their peers' changing moods and psychological states during the interaction. While these changes cannot be measured directly, they affect physiological changes which can be measured as physiological signals such as heart rate and skin conductance. Recently, portable and lightweight electronic measurement devices have allowed researchers to measure such signals during an interactive experiment without interfering significantly with the experiment's progress. Using these signal to gain deeper understanding of group interaction is a major current effort of the research community. 


\section{Physiological synchrony}
\subsection{Synchrony}
Notionally, synchrony occurs when the development in time of the physiological states of two (or more) individuals, align in a way that implies a common causal factor and matched phase. 
While the concept itself is considered intuitively clear, defining it quantitatively has proven less straightforward. As shall be explained in the next two sections, the main reason is the fact that physiological time series tend to exhibit complex patterns of auto-dependence. 

\subsection{Quantifying synchrony}
Typically, to demonstrate synchrony, the researcher first attempts to model the dynamics of each time series separately. For example, applying short-memory and linearity assumptions leads to ARIMA-type models. Once the model for each signal's auto-dependence is specified, its parameters for the particular signal are estimated. Then, each sample (or short batch of samples) can be predicted based on past samples. Any deviation of the actual signal from model prediction is assumed to be due to external (i.e. neural) influences on the physiological system. In some contexts, this is referred to as the \emph{innovation component} of the signal. 

Taking the innovation component of each participant's signal, the researcher then looks for indications of dependence between the signals of individuals that were in social interaction. A widely-adopted practice is to demonstrate that the dependence between signals of individuals that actually interacted is significantly higher than the (chance) dependence between randomly-matched pairs of signal.

\subsection{Difficulties in quantifying synchrony}
The main difficulty associated with the above approach is in modeling the self-dependence of a physiological signal. Physiological signals have been shown to exhibit complex patterns of long-term auto-dependence. This makes parameter estimation for such models difficult. In some cases, the available sample duration is not much longer than the desired maximum auto-dependence interval, which renders the approach all but unfeasible. A plethora of competing methods have been applied to the problem. Different design choices lead to different results.

Then, the author usually applies a second model, for how the signal from one individual may affect the signal from the other individual. This model is used to test whether dependence exists across individuals. 

To illustrate the analysis approach outlined above, let us follow the steps required, using a simple example where an interaction is performed in pairs, with the RR-interval of each individual recorded during the interaction. First, a model of each signal's self-dependence is applied. The part of the signal that is explained by the model is subtracted, leaving the unexplained part, sometimes referred to as the instantaneous innovation. The time-series of instantaneous innovations are analyzed for cross-dependence, e.g. via linear regression or lagged regression. In order to demonstrate synchrony that is induced by the interaction, the researcher typically compares the quality of fit achieved when using one member's signal to predict his or her true pair's signal, vs. the quality of fit when matching signals randomly (i.e. signal of individuals from distinct pairs).



\subsection{Heart Rate and Skin Conductivity}
The heart rate is measured by analyzing the electric potential at electrodes placed on the subject's torso, and is also known as the RR-interval (a name derived from the "R" peak in the ECG), or IBI (for inter-beat interval). The variation of heart rate over time in healthy individuals is controlled by a few different pathways in the autonomous nervous system. In many different settings, persons in direct social contact (i.e. in physical proximity and exchanging social cues) have been observed to have some level of synchrony in their heart rates over the course of an experiment, which is understood as an affirmation of the fact that the regulatory states of the two individuals are both affected by (or respond to) the cues exchanged during their social interaction.

Skin conductivity varies mainly due to the presence of sweat, and is measured in units of [ohm\slash meter].

\section{Experimental Setup}
A brief description of the experiment on which this work focuses follows. The experiment was repeated about 50 times, with distinct teams of three members each.
\begin{itemize}
    \item Each team's interaction was subdivided into 4 sessions:
    \begin{itemize}
        \item In the first and last sessions, team members were simply asked to relax. These are termed baseline sessions.
        \item In the second sessions, team members played the drums together for several minutes, accompanied by a background tempo (teams were randomly assigned on of two types of tempos)
        \item In the third session, the team played the drums again, this time without any background tempo (this is termed the freestyle session) 
    \end{itemize}
    \item Before the interaction started, each participant had to answer a personality questionnaire which focused on social affinity.
    \item Just before the interaction started, the participants filled another questionnaire, regarding their current mood, sleep, and caffeine intake.
    \item Between the two interaction sessions, the participants filled a questionnaire to evaluate the effect of the interaction on their moods.
    \item Respondents with musical background were not admitted.
\end{itemize}

\subsection{Literature Survey}
The existence of synchrony in IBI signals has been established in various situations and setups. More recent papers in this field, deal with the dynamics of synchrony, i.e. analyzing periodic patterns underlying synchrony and devising synchrony-finding methods that can be used viably over short time frames (e.g. minutes or even sub-minutes), in order to analyze the progression of synchorny over the course of an experiment. 




\section{Research Goals}
Our goals in this research are as follows: 
\begin{itemize}
\item Link between quantitative measures of synchrony and the individual ("subjective") experience during the experiment
\item Investigate how the unpredictable background beat interacted with the formation of synchrony
\item Investigate the dynamics of three-person synchrony from a social perspective.
\end{itemize}


\end{document}

