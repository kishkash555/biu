%%%%%%%%%%%%%%%%%%%%%%%%%%%%%%%%%%%%%%%%%%%%%%%%%%%%%%%%%%%%%%%%%%%%%%%%%%%%%%%%
%2345678901234567890123456789012345678901234567890123456789012345678901234567890
%        1         2         3         4         5         6         7         8

%\documentclass[letterpaper, 10 pt, conference]{ieeeconf}  % Comment this line out
                                                          % if you need a4paper
\documentclass[a4paper, 11pt]{article}      % Use this line for a4
                                                          % paper

\usepackage[utf8]{inputenc}
\usepackage[T1]{fontenc}

% The following packages can be found on http:\\www.ctan.org
%\usepackage{graphics} % for pdf, bitmapped graphics files
%\usepackage{epsfig} % for postscript graphics files
%\usepackage{mathptmx} % assumes new font selection scheme installed
%\usepackage{mathptmx} % assumes new font selection scheme installed
%\usepackage{amsmath} % assumes amsmath package installed
%\usepackage{amssymb}  % assumes amsmath package installed

\title{\LARGE \bf
Analysis of synchronization among individuals
}

\author{Shahar Siegman}

\begin{document}



\maketitle
\thispagestyle{empty}
\pagestyle{empty}


%%%%%%%%%%%%%%%%%%%%%%%%%%%%%%%%%%%%%%%%%%%%%%%%%%%%%%%%%%%%%%%%%%%%%%%%%%%%%%%%
\begin{abstract}

A research proposal.

\end{abstract}

\section{Introduction}
Team performance is an important research topic in social sciences. Successful teams perform better while their members experience more positive feelings, compared to less successful teams. This is why it's not surprising that team success is determined not only by each individual team member's competence towards the task, but also by the extent to which team members are able to respond effectively to their peers' changing moods and psychological states during the interaction.

While changes in moods and psychological states cannot be measured directly, changes in physiological signals such as heart rate and skin conductance can serve as proxies. Recently, portable and lightweight electronic measurement devices have enabled researchers to measure their subjects' physiology during an interactive experiment without interfering significantly with the experiment's progress. This opens up the opportunity for a deeper understanding of the evolution of team members' psychological state during the course of an experiment and how different modes of interaction relate to different indicators of team success.


course
The introduction of portable electronic equipment capable of measuring physiological signals    
in order to perform well and establish a positive experience for each team member.



team success is determined in a large part by how The  interactions among team members play an important role in team success, beyond the effect of individual capabilities. More recently, experiments 
skilSome teams perform tasks better while their members have more positive experiences, even when the individual memb 



\subsection{Physiological synchrony}
Physiological synchorny between individuals can be roughly defined as any form of statistical relationship between any two (or more) physiological signals collected concurrently from those individuals during an interaction. In the word ``signal'', we refer to any time-based measurement representing some aspect of their current physiology.


While many types of signals fall into the above definition, most research has focused on a few types of signals whose regulatory significance is relatively well-understood and are easy to collect. We will briefly cover the two types of signals that are used in the current research: heart rate and skin conductivity.

\subsection{Heart Rate and Skin Conductivity}
The heart rate is measured by analyzing the electric potential at electrodes placed on the subject's torso, and is also known as the RR-interval (a name derived from the "R" peak in the ECG), or IBI (for inter-beat interval). The variation of heart rate over time in healthy individuals is controlled by a few different pathways in the autonomous nervous system. In many different settings, persons in direct social contact (i.e. in physical proximity and exchanging social cues) have been observed to have synchronous variation of their RR-interval over the course of the experiment, which is understood as an affirmation of the fact that the regulatory states of the two individuals are both affected by (or respond to) the cues exchanged during their social interaction.

Skin conductivity varies mainly due to the presence of sweat, and is measured in units of [ohm\slash meter].

\subsection{Literature Survey}
The existence of synchrony in IBI signals has been established in various situations and setups. More recent papers in this field, deal with the dynamics of synchrony, i.e. analyzing periodic patterns underlying synchrony and devising synchrony-finding methods that can be used viably over short time frames (e.g. minutes or even sub-minutes), in order to analyze the progression of synchorny over the course of an experiment. 

\section{Research Goals}
Our goals in this research are as follows: 

\textbf{From the methodological perspective:}
\begin{itemize}
\item Understanding the applicability of some of the common methods for testing (and quantifying) synchrony for our particular data-set.
\item Establishing the trade-off between effect significance and analysis time-frame for our data-set.
\item Using periodic function analysis to characterize the typical delays associated with synchronous behavior in the different types of signals.   
\end{itemize}

\textbf{From the physio-psychological perspective:}
\begin{itemize}
\item Link between quantitative measures of synchrony and the individual ("subjective") experience during the experiment
\item Investigate how the unpredictable background beat interacted with the formation of synchrony
\item Investigate the dynamics of three-person synchrony from a social perspective.
\end{itemize}


\end{document}

