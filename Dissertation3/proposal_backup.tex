%%%%%%%%%%%%%%%%%%%%%%%%%%%%%%%%%%%%%%%%%%%%%%%%%%%%%%%%%%%%%%%%%%%%%%%%%%%%%%%%
%2345678901234567890123456789012345678901234567890123456789012345678901234567890
%        1         2         3         4         5         6         7         8

%\documentclass[letterpaper, 10 pt, conference]{ieeeconf}  % Comment this line out
                                                          % if you need a4paper
\documentclass[a4paper, 11pt]{article}      % Use this line for a4
                                                          % paper

\usepackage[utf8]{inputenc}
\usepackage[T1]{fontenc}
\usepackage{csquotes}

% The following packages can be found on http:\\www.ctan.org
%\usepackage{graphics} % for pdf, bitmapped graphics files
%\usepackage{epsfig} % for postscript graphics files
%\usepackage{mathptmx} % assumes new font selection scheme installed
%\usepackage{mathptmx} % assumes new font selection scheme installed
%\usepackage{amsmath} % assumes amsmath package installed
%\usepackage{amssymb}  % assumes amsmath package installed

\title{\LARGE \bf
Investigation of team performance 
and physiological synchrony
using deep learning }

\author{Shahar Siegman}

\begin{document}



\maketitle
\thispagestyle{empty}
\pagestyle{empty}


%%%%%%%%%%%%%%%%%%%%%%%%%%%%%%%%%%%%%%%%%%%%%%%%%%%%%%%%%%%%%%%%%%%%%%%%%%%%%%%%
\begin{abstract}

A research proposal.

\end{abstract}

\section{Background}
\subsection{Physiological synchrony}
\subsubsection{Concept}
Notionally, synchrony occurs when the development in time of the physiological states of two (or more) individuals, align in a way that implies a common causal factor and matched phase (\cite{palumbo2017interpersonal}). 
While the concept itself is considered intuitively clear, defining it quantitatively has proven less straightforward. As shall be explained in the next two subsections, the main reason is the fact that physiological time series tend to exhibit complex patterns of auto-dependence \cite{ivanov20011}. 

\subsubsection{Quantifying synchrony}
Typically, to demonstrate synchrony, the researcher first attempts to model the dynamics of each time series separately. While such a model has no direct significance in the context of investigating synchrony, it helps establish the  baseline dynamics of a "stable" signal i.e. the expected development of a signal that is not influenced by other participants' emotions. As an example for a model of a stable signal, consider the ARIMA family of models. Such models assume short-memory, linearity and stationarity. 

Once the model for each signal's auto-dependence is specified, its parameters for the particular participant are estimated. This model is used to separate the interaction-related component as follows: each sample (or short batch of samples) is first predicted based on past samples. Deviations of the actual signal from model prediction are assumed to be due to external influences, including neural influence stemming from psychological changes. In some time-series modeling contexts, this is referred to as the \emph{innovation component} of the signal. 

Taking the innovation component of each participant's signal, the researcher then quantifies the level of dependence between the individuals' signals. A widely-adopted practice is to compare the empirical level of statistical dependence between the population of "true pairs" (i.e. the pairs of signals from individuals that actually interacted) to the level found in randomly-matched pairs (signals of individuals from different groups that were matched at random post-experiment).

\subsubsection{Difficulties in quantifying synchrony}
The main difficulty associated with the above approach is in modeling the self-dependence of a physiological signal. Physiological signals have been shown to exhibit complex patterns of long-term auto-dependence, and do not generally exhibit stationarity. This makes ARIMA a poor choice, and more elaborate models require more parameters whose estimation accuracy may be poor. In most cases, the available sample duration is too short for estimating the large number of parameters that a long-range autocorrelation model would require. Many different methods have been applied to the problem, and the choice of methodology can potentially lead to different conclusions regarding synchrony.

\subsection{Physiological synchrony in the study of team performance}
\subsubsection{Team performance}
Team performance is an important research topic in social sciences. Team performance has been linked to better individual (subjective) experiences during the interaction \cite{lodahl1961psychometric}, and also to group cohesiveness -- the desire of team members to be part of the group (\cite{cartwright1968nature}). However, none of the models proposed over the years have prevailed. Differences in experimental and observational setups, as well as different analysis approaches, have restricted the ability to generalize and recreate findings across setups (\cite{beal2003cohesion}). Modeling team performance is therefore still a matter of active research, see for example \cite{collins2019explorations}.

\cite{akinola2010measuring} Have suggested that physiological measures be incorporated into the (metaphoric) toolbox of the organizational researcher as a way to \enquote{deepen theoretical insights and enrich our understanding of human behavior in organizations}. \cite{chikersal2017deep} have demonstrated correlation between shared arousal, as measured by electrodermal activity, and group satisfaction. \cite{danyluck2018intergroup}, have taken a different approach, by casting synchrony as the outcome, and examining which of the different priming level induces a higher level of synchrony. 

\section{The experimental setup}

\subsection{Heart Rate and Skin Conductivity}
The heart rate is measured by analyzing the electric potential at electrodes placed on the subject's torso. The RR-interval (a name derived from the "R" peak in the ECG), or IBI (for inter-beat interval) are both names for the interval between successive beats, in other words, the instantaneous reciprocal of the heart rate. The variation of heart rate over time in healthy individuals is controlled by several different pathways in the autonomous nervous system. In many different settings, persons in direct social contact (i.e. in physical proximity and exchanging social cues) have been observed to have some level of synchrony in their heart rates over the course of an experiment, which is understood as an affirmation of the fact that the regulatory states of the two individuals are both affected by (or respond to) the cues exchanged during their social interaction.

Skin conductivity varies mainly due to the presence of sweat. It is measured in units of [ohm\slash meter]. Its level is related to sympathetic nervous system activity.

\section{Experiment itinerary and details of data recorded}
A brief description of the experiment on which this work focuses follows. The experiment was repeated about 50 times, with distinct teams of three members each.
\begin{itemize}
    \item Each team's interaction was subdivided into 4 sessions:
    \begin{itemize}
        \item In the first session, team members were simply asked to relax. This is the first of two \emph{baseline} sessions.
        \item In the second session, team members played the drums together for several minutes, accompanied by a background tempo (teams were randomly assigned one of two types of tempos)
        \item In the third session, the team played the drums again, this time without any background tempo (this is termed the freestyle session)
        \item In the last session, the team members again were in the same room without performing any task. This is the second baseline session.
    \end{itemize}
    \item Before arriving, each participant had to fill in and submit a personality questionnaire which focuses on social affinity.
    \item Just before the interaction started, the participants filled another questionnaire, regarding their current mood, sleep, and caffeine intake.
    \item Between the two interaction sessions, the participants filled a questionnaire to evaluate the effect of the interaction on their moods.
    \item Participants were pre-screened for musical  background. Respondents with significant musical experience were not admitted.
\end{itemize}



\section{Proposed Research}

\subsection{Focus of the research}
Each participant's experience has both physiological and psychological perspectives. The physiological perspective is represented in this experiment by the recorded physiological signals. The psychological perspective is represented by the questionnaire responses provided before and after the group drumming session. The existence of interrelations between psychology and the signals measured in this experiment is already well established. Likewise, substantial attention has been given to demonstrating that physiological signals from different individuals can exhibit similar patterns during social interactions - this is the phenomenon known as physiological synchrony. However, the interplay between physiological synchrony and group performance is not as well understood, and will be the focus of the current research.

\subsection{Deep learning techniques for investigating dependence between parameters and their application}

Machine-learning (ML) models are very flexible in terms of inputs and outputs. They can also be adapted, through specialized cost functions, to learn representations with certain desirable features. These qualities will allow us to uncover relationships among the group participants' signals, and achieve greater insight about the phenomenon of physiological synchronization, its occurrence in group interaction, and its relation to group performance indicators.

The following two paragraphs describe approaches used in the literature to uncover dependencies in data, and suggestions how they may be adapted to our case.

\subsubsection{Input masking}
Deep learning models are known to exploit features in input space without requiring those features to be to explicitly specified to them. An effective generalization is achieved when the model learns to utilize relevant information while ignoring clutter. This has guided [] to propose \emph{masking} as a way to investigate the dependence between input and output spaces and deduce relevance. In short, the technique calls for replacing a consistent segment of the input samples with non-informative data, then quantify the change in model accuracy between the mutilated samples and the original sample. If a significant change occurs following the substitution, it's an indication that the segment of the input that has been masked was relevant for the output.

Taking this approach to our context, we can test to see how much the signals from two of a group participants are helpful in predicting the psychological signals of the third. This can be done by first training a network to predict one of the participant's psychological indices based on the three signals, then comparing the result between actual test groups and surrogate test groups where the two auxiliary participants' signals were replaced with signals from other groups.


\subsubsection{Deep canonical correlation analysis}
Deep canonical correlation analysis \cite{andrew2013deep} is concerned with creating the best representation for the entire input space, when fed with only a subset of the input coordinates. The DCCA network training pattern is akin to other encoders, but the architecture splits the input vector at a predetermined location and processes each part separate. Therefore, the representation layer contains distinct groups of neurons, each can be attributed to one side of the split in the input space. Additionally, the output neurons are trained to be \emph{minimally correlated within} each group, and \emph{maximally correlated (in pairs) between the groups}. This means that if one side of the split is missing in the input, its representation can be estimated simply by copying the representation of the correlated neurons from the other side.

For our purposes, DCCA can be trained with the signals from one individual in the group on one side of the split and another group member's signals on the other side. Comparing the reconstruction fidelity of the encoder-decoder system, when both signals are present, and then when one of the signals is missing, should provide interesting insight regarding the features that are common between group participants' signals.



\bibliographystyle{apalike}
\bibliography{ref}
\end{document}

