%%%%%%%%%%%%%%%%%%%%%%%%%%%%%%%%%%%%%%%%%%%%%%%%%%%%%%%%%%%%%%%%%%%%%%%%%%%%%%%%
%2345678901234567890123456789012345678901234567890123456789012345678901234567890
%        1         2         3         4         5         6         7         8

%\documentclass[letterpaper, 10 pt, conference]{ieeeconf}  % Comment this line out
                                                          % if you need a4paper
\documentclass[a4paper, 11pt]{article}      % Use this line for a4
                                                          % paper

\usepackage[utf8]{inputenc}
\usepackage[T1]{fontenc}

% The following packages can be found on http:\\www.ctan.org
%\usepackage{graphics} % for pdf, bitmapped graphics files
%\usepackage{epsfig} % for postscript graphics files
%\usepackage{mathptmx} % assumes new font selection scheme installed
%\usepackage{mathptmx} % assumes new font selection scheme installed
%\usepackage{amsmath} % assumes amsmath package installed
%\usepackage{amssymb}  % assumes amsmath package installed

\title{\LARGE \bf
Team performance and physiological synchrony
}

\author{Shahar Siegman}

\begin{document}



\maketitle
\thispagestyle{empty}
\pagestyle{empty}


%%%%%%%%%%%%%%%%%%%%%%%%%%%%%%%%%%%%%%%%%%%%%%%%%%%%%%%%%%%%%%%%%%%%%%%%%%%%%%%%
\begin{abstract}

A research proposal.

\end{abstract}

\section{Background}
\subsection{General - team performance}
Team performance is an important research topic in social sciences. Successful teams perform better while their members experience more positive feelings, compared to less successful teams. This is why it's not surprising that team success is determined not only by each individual team member's competence towards the task, but also by the extent to which team members are able to respond effectively to their peers' changing moods and psychological states during the interaction.

While changes in moods and psychological states cannot be measured directly, changes in physiological signals such as heart rate and skin conductance can serve as proxies. Recently, portable and lightweight electronic measurement devices have enabled researchers to measure their subjects' physiology during an interactive experiment without interfering significantly with the experiment's progress. The extra data collected in this way, creates an opportunity for a deeper understanding of the evolution of team members' psychological state during the course of an experiment, and, importantly, how this evolution relates to team success.


\subsection{Physiological synchrony}
The seemingly intuitive notion of synchrony seems to lack a more concrete definition, at least one that is widely accepted. Notionally, it is when the physiological state of two (or more) individuals, develops in a way that implies a common causal factor. Usually, the researcher would strive to neutralize any external influences, in order to investigate how much, and in what ways, synchrony arises as the team members interact, sense and respond to each other. In my research, I will regard any form of statistical dependence between any two (or more) physiological signals collected concurrently from the studied individuals during their interaction, as an indication of physiological synchrony.

\subsection{Heart Rate and Skin Conductivity}
The heart rate is measured by analyzing the electric potential at electrodes placed on the subject's torso, and is also known as the RR-interval (a name derived from the "R" peak in the ECG), or IBI (for inter-beat interval). The variation of heart rate over time in healthy individuals is controlled by a few different pathways in the autonomous nervous system. In many different settings, persons in direct social contact (i.e. in physical proximity and exchanging social cues) have been observed to have some level of synchrony in their heart rates over the course of an experiment, which is understood as an affirmation of the fact that the regulatory states of the two individuals are both affected by (or respond to) the cues exchanged during their social interaction.

Skin conductivity varies mainly due to the presence of sweat, and is measured in units of [ohm\slash meter].

\section{Experimental Setup}
A brief description of the experiment on which this work focuses follows. The experiment was repeated about 50 times, with distinct teams of three members each.
\begin{itemize}
    \item Each team's interaction was subdivided into 4 sessions:
    \begin{itemize}
        \item In the first and last sessions, team members were simply asked to relax. These are termed baseline sessions.
        \item In the second sessions, team members played the drums together for several minutes, accompanied by a background tempo (teams were randomly assigned on of two types of tempos)
        \item In the third session, the team played the drums again, this time without any background tempo (this is termed the freestyle session) 
    \end{itemize}
    \item Before the interaction started, each participant had to answer a personality questionnaire which focused on social affinity.
    \item Just before the interaction started, the participants filled another questionnaire, regarding their current mood, sleep, and caffeine intake.
    \item Between the two interaction sessions, the participants filled a questionnaire to evaluate the effect of the interaction on their moods.
    \item Respondents with musical background were not admitted.
\end{itemize}

\subsection{Literature Survey}
The existence of synchrony in IBI signals has been established in various situations and setups. More recent papers in this field, deal with the dynamics of synchrony, i.e. analyzing periodic patterns underlying synchrony and devising synchrony-finding methods that can be used viably over short time frames (e.g. minutes or even sub-minutes), in order to analyze the progression of synchorny over the course of an experiment. 




In the word ``signal'', we refer to any time-based measurement representing some aspect of their current physiology.


While many types of signals fall into the above definition, most research has focused on a few types of signals whose regulatory significance is relatively well-understood and are easy to collect. We will briefly cover the two types of signals that are used in the current research: heart rate and skin conductivity.


\section{Research Goals}
Our goals in this research are as follows: 

\textbf{From the methodological perspective:}
\begin{itemize}
\item Understanding the applicability of some of the common methods for testing (and quantifying) synchrony for our particular data-set.
\item Establishing the trade-off between effect significance and analysis time-frame for our data-set.
\item Using periodic function analysis to characterize the typical delays associated with synchronous behavior in the different types of signals.   
\end{itemize}

\textbf{From the physio-psychological perspective:}
\begin{itemize}
\item Link between quantitative measures of synchrony and the individual ("subjective") experience during the experiment
\item Investigate how the unpredictable background beat interacted with the formation of synchrony
\item Investigate the dynamics of three-person synchrony from a social perspective.
\end{itemize}


\end{document}

